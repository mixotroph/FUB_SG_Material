\documentclass[11pt,a4paper,DIV=10,parskip=half,BCOR=0mm]{scrartcl}
\usepackage[utf8]{inputenc}
\usepackage[T1]{fontenc}
\usepackage[ngerman]{babel}
\usepackage{scrpage2} % pagestyle
\usepackage{geometry} % Seitenränder
\usepackage{amsmath}
\usepackage{amsfonts}
\usepackage{amssymb}
\usepackage{amsthm}
\usepackage{multicol}
\usepackage{graphicx}
\usepackage{float}
\usepackage{marginnote} % Randnotizen
\usepackage{tabularx} % Tabelle mit spaltentyp 'X' für feste breite
\usepackage{fancybox}

% Seitenränder einstellen: marginparwidth wichtig für Randnotizen!
\geometry{a4paper, top=25mm, left=35mm, right=50mm, bottom=30mm,
	headsep=10mm, footskip=12mm, marginparwidth=45mm, marginparsep=3mm}

% Fußzeile
\pagestyle{scrheadings}
\ifoot{ \hrule {\textcopyright} \ C. van Heteren-Frese}
\begin{document}
\pagenumbering{gobble}
%
% === HEADER ====
\setlength{\tabcolsep}{0mm} % kein Innenrand bei Spaleten
\begin{tabularx}{\linewidth}{lXr}
{\LARGE\textbf{\textsf{AB 1: Interessengruppen des Smartgrid}}} & &%\includegraphics[scale=0.04]{images/oeko}
\\
\hline
\end{tabularx}
% === HEADER END ===
%
\section*{Rollenspiel}
Ob\marginnote{\includegraphics[scale=0.25]{images/rollenspiel}} das Smart Grid wirklich eine gute Sache ist, hängt stark von dem jeweiligen Standpunkt
verschiedener Interessengruppen ab: Was einige gut finden, halten andere vielleicht für eine
schlechte Idee! Um einen groben Überblick über die möglichen Sichtweisen zu bekommen,
sollen drei unterschiedliche Positionen in einem Rollenspiel dargestellt werden.
\subsection*{Talkshow}
Stellt euch vor, ihr seid in eine beliebte Talk Show eingeladen, um den Zu\-schau\-ern
die Meinung eurer Gruppe zum Thema Smart Grid zu erläutern und -- falls nötig -- zu verteidigen. Es schauen sehr viele Menschen zu. Deshalb ist es sehr wichtig, dass ihr überzeugend rüberkommt und eure Widersacher und auch das Publikum überzeugt.
\subsection*{Ablauf}
Es werden drei Gruppen gebildet. Jede Gruppe spielt eine erfundene Interessengruppe (z.B. privater Verbraucher, Erzeuger, Softwarehersteller). Alle wichtigen Informationen zur jeweiligen Gruppe stehen auf einem separaten Informationsblatt.
\subsection*{Ziel}
Es soll ein Streitgespräch entstehen, in dem jede Gruppe ihre eigenen
Standpunkte und Argumente vorträgt und denen der anderen Gruppen
begründet wiederspricht. Der Sieger bekommt eine Tafel Schokolade! \\
\hrule
\section*{Aufgaben}
\subsection*{Aufgabe 1}
Lese \marginnote{\includegraphics[scale=0.25]{images/task}} dir zuerst Deine Ausgangssituation alleine und in Ruhe durch. Überlege dir im Anschluss mit den Mitgliedern deiner Gruppe, welche Argumente für eure Position sprechen. Schreibt euch drei Argumente, die
Ihr vortragen wollt und die ihr für besonders wichtig haltet, auf. 
\subsection*{Aufgabe 2}
Nach der Vorbereitungsphase geht’s los! Jetzt seid Ihr ca. 10 Minuten auf Sendung. Der Moderator stellt euch am Anfang ein paar Fragen, aber ihr dürft und sollt einfach dazwischen quatschen, wenn sich die Argumente
der anderen nicht mit euren eigenen übereinstimmten.
\end{document}